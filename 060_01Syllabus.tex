\documentclass[12pt]{article}
\textwidth=7in
\textheight=9.5in
\topmargin=-1in
\headheight=0in
\headsep=.5in
\hoffset  -.85in

\pagestyle{empty}
\usepackage{hyperref}

\renewcommand{\thefootnote}{\fnsymbol{footnote}}
\begin{document}

\begin{center}
{\bf MATH 060,\ Sec. 01, \ ID \# 000712, \ MWF 8:30am - 9:20am, \\ Room:  Bellows Academic Center 232  
}
\end{center}

\setlength{\unitlength}{1in}

\begin{picture}(6,.1) 
\put(0,0) {\line(1,0){6.25}}         
\end{picture}

 

\renewcommand{\arraystretch}{2}

\noindent\textbf{Instructor:} Dr. Matthew Zabka,  \textbf{Office:} Science and Math 227, \textbf{Phone:} 537-6056 \\
\textbf{E-Mail:} \href{mailto:matthew.zabka@smsu.edu}{matthew.zabka@smsu.edu}

\vskip.15in
\noindent\textbf{Office Hours:} MW 11:30am - 3:00pm, T 9:00am-10:30am and 1:00pm-2:30pm, F 11:30-12:30, and by appointment

\vskip.15in

\noindent\textbf{Textbook:} \textit{Essentials of Elementary and Intermediate Algebra A Combined Approach},\\  \textbf{Edition:} Second \textbf{Author:} Charles P. McKeague, \textbf{ISBN:} 978-1-63098-101-3 

\vskip.15in
\noindent\textbf{Prerequisites:} This is a remedial class, but a student should should still be familiar with arithmetic operations on real numbers, exponents, and radicals.

\vspace*{.15in}

\noindent \textbf{Tenative Course Outline:} 

\begin{center} \begin{minipage}{5in}
\begin{flushleft}
Chapter 0 	\dotfill	1 weeks\\
Chapter 1 	\dotfill	3 weeks\\
Chapter 2 	\dotfill	2 weeks \\
First Exam	\dotfill	5. October\\
Chapter 3	\dotfill	2 weeks\\
Chapter 4 	\dotfill 	3 weeks \\
Second Exam	\dotfill	16. November\\
Chapter 5	\dotfill	3 weeks\\
Chapter 6 	\dotfill 	2 weeks \\
Final Exam 	\dotfill 	17. December 2018, 4:00pm - 5:50pm
\end{flushleft}
\end{minipage}
\end{center}

\vspace*{.15in}

\noindent \textbf{Course Description}:  Algebraic skill-building for students anticipating further courses in mathematics or areas using mathematics. Covers polynomials, roots and powers, lines and solving linear inequalities, and linear, quadratic, and rational equations.

\vspace*{.15in}
\noindent \textbf{Learning Outcomes}: Upon completion of this course students will have developed the algebraic knowledge and skills to successfully complete MATH 110: College Algebra.

\vskip.25in
\noindent\textbf{Grading}: Grades are determined as follows:\\
\begin{center} \begin{minipage}{5in}
\begin{flushleft}
Two in-class exams \dotfill 40\%\\
Final exam \dotfill 40\%\\
In-class worksheets \dotfill 10\%\\
Homework \dotfill 10\%\\

\vskip.15in

A \dotfill 93\% - 100 \%\\
A- \dotfill 90\% - 92.9\%\\
B+ \dotfill 87\% - 89.9\%\\
B  \dotfill 83\% - 86.9\%\\
and so on
\end{flushleft}
\end{minipage}
\end{center}

\vskip.15in
\noindent\textbf{Homework and Worksheets}: The best way to learn mathematics is to do mathematics. Homework will therefore be assigned after every class.  Students will use the online homework system \href{https://www.xyzhomework.com/}{\underline{xyzhomework}}. No late homework will be accepted. In-class worksheets must be done in class. Your two lowest homework scores, as well as your two lowest worksheet score, will be dropped.

\vskip.15in
\noindent\textbf{Exams}: There will be two in-class exams. Only in the case of an extreme emergency, religious conflict or sanctioned university event will a makeup exam be considered. In the case of an emergency, you must contact me as soon as possible to make me aware of the emergency. If your final exam percentage is higher than your worst in-class exam's percentage, then your final exam percentage will replace your worst in-class exam's percentage. In the case of a religious conflict or sanctioned university event, or should you require a nontraditional testing environment, you must notify me at least one week before the exam.

\vskip.15in
\noindent\textbf{Final Exam}: The final exam will be held on Monday, 17. December, 2018, 4:00pm - 5:50pm.  It is cumulative. Any student who fails to earn 60 \% or better on the final exam will receive a course grade no higher than a D+.

\vskip.15in
\noindent\textbf{Academic Honesty}:  All students are expected to adhere to the SMSU Code of Conduct. Any type of academic dishonesty will not be tolerated. In the event of academic dishonesty, the student will receive an F in the course.

\vskip.15in
\noindent\textbf{Extra Help}:  Do not hesitate to come to my office during office hours or by appointment. The SMSU Math Help Center, which offers free tutoring, is also available to you. The Math Help Center is located in the Academic Commons, IL224.


\vskip.15in
\noindent\textbf{Attendance Policy}: Students are expected to attend classes regularly.  Attendance will occasionally be taken. Should you miss class, it is your responsibility to obtain a copy of the notes from a classmate. You are responsible for the material that was covered and any announcements that were made.

\vskip.15in
\noindent\textbf{Calculators}: Calculators may not be used on exams or on worksheets.

\vskip.15in
\noindent\textbf{D2L}: Announcements and updates to this syllabus can be found on our course's D2L website.


\vskip.15in
\noindent\textbf{Important Dates}:
\begin{center} \begin{minipage}{5in}
\begin{flushleft}
Last date to drop with full refund 	\dotfill 31. August \\
Last date to drop with a W 			\dotfill 27. November\\
No class (Labor Day)				\dotfill 03. September\\
No class (Fall Break)               \dotfill 15-16. October \\
No class (Veterans Day)				\dotfill 12. November\\ 
No class (Thanksgiving)				\dotfill 21-23. November\\
Final Exam 							\dotfill 17. December, 2018, 4:00pm - 6:50pm\\
\end{flushleft}
\end{minipage}
\end{center}

\vskip.15in
\noindent \textbf{Disclaimer:} Course material may be used anonymously for assessment of program and liberal education program student learning outcomes. 

\end{document}