\documentclass[12pt]{article}
\textwidth=7in
\textheight=9.5in
\topmargin=-1in
\headheight=0in
\headsep=.5in
\hoffset  -.85in

\pagestyle{empty}
\usepackage{hyperref}

\renewcommand{\thefootnote}{\fnsymbol{footnote}}
\begin{document}

\begin{center}
{\bf MATH 150,\ Sec. 01, \ ID \# 000698, \ MTWThF 10:30am - 11:20am, \\ Room:  Bellows Academic Center 232  
}
\end{center}

\setlength{\unitlength}{1in}

\begin{picture}(6,.1) 
\put(0,0) {\line(1,0){6.25}}         
\end{picture}

 

\renewcommand{\arraystretch}{2}

\noindent\textbf{Instructor:} Dr. Matthew Zabka,  \textbf{Office:} Science and Math 227, \textbf{Phone:} 537-6056 \\
\textbf{E-Mail:} \href{mailto:matthew.zabka@smsu.edu}{matthew.zabka@smsu.edu}

\vskip.15in
\noindent\textbf{Office Hours:} MW 11:30am - 3:00pm, T 9:00am-10:30am, F 11:30-12:30, and by appointment

\vskip.15in
\begin{center}
\underline{\textbf{Course Textbooks}}\\
\end{center}
\noindent\textbf{Main Textbook:} \textit{Calculus: Early Transcendentals, 8th Edition}, \textbf{Author:} James Stewart, \textbf{ISBN:} 978-1-285741-55-0\\
\noindent\textbf{See Also:} \textit{Calculus Volume 1}, \textbf{Author:} OpenStax, \textbf{ISBN:} 978-1-938168-02-4 \\
\textbf{Website:} \url{https://openstax.org/details/books/calculus-volume-1}\\
Homework will be assigned out of the OpenStax textbook.

\vskip.15in
\noindent\textbf{Prerequisites:} Math 125 or Math 135 or three years of high school mathematics, including trigonometry. If you do not meet these requirements, you must drop the course. If you are unsure if you meet these requirements, please speak with me.

\vspace*{.15in}

\noindent \textbf{Tenative Course Outline:} 

\begin{center} \begin{minipage}{5in}
\begin{flushleft}
P1, P2, P3, and P4 \dotfill 1 day\\
Chapter 1 	\dotfill 2 weeks\\
Chapter 2 	\dotfill 2 weeks\\
First Exam 	\dotfill  28. September\\
Chapter 3 	\dotfill 3 weeks \\
Second Exam \dotfill 19. October\\
Chapter 4 	\dotfill 3 weeks \\
Third Exam 	\dotfill 16. November\\
Chapter 5 	\dotfill 3 weeks \\
Final Exam 	\dotfill 18. December 2018, 10:00am - 11:50am
\end{flushleft}
\end{minipage}
\end{center}

\vspace*{.15in}

\noindent \textbf{Course Description}:  A detailed study of the mathematics needed for calculus. Concepts are presented and explored from symbolic, graphical, and numerical perspectives. Basic concepts covered include polynomial, rational, exponential, logarithmic, and trigonometric functions, complex numbers, linear systems, numerical patterns, sequences and series.

\vskip.25in
\noindent\textbf{Grading}: Grades are determined as follows:\\
\begin{center} \begin{minipage}{5in}
\begin{flushleft}
Three in-class exams \dotfill 45\%\\
Final exam \dotfill 35\%\\
In-class worksheets \dotfill 10\%\\
Homework \dotfill 10\%\\

\vskip.25in

A \dotfill 93\% - 100 \%\\
A- \dotfill 90\% - 92.9\%\\
B+ \dotfill 87\% - 89.9\%\\
B  \dotfill 83\% - 86.9\%\\
and so on
\end{flushleft}
\end{minipage}
\end{center}

\vskip.25in
\noindent\textbf{Homework and Worksheets}: The best way to learn mathematics is to do mathematics. Homework will therefore be assigned regularly. No late homework will be accepted. In-class worksheets must be done in class. Your two lowest homework scores, as well as your two lowest worksheet scores, will be dropped.

\vskip.25in
\noindent\textbf{Exams}: There will be three in-class exams. Only in the case of an extreme emergency, religious conflict or sanctioned university event will a makeup exam be considered. In the case of an emergency, you must contact me as soon as possible to make me aware of the emergency. If your final exam percentage is higher than your worst in-class exam's percentage, then your final exam percentage will replace your worst in-class exam's percentage. In the case of a religious conflict or sanctioned university event, or should you require a nontraditional testing environment, you must notify me at least one week before the exam.

\vskip.25in
\noindent\textbf{Final Exam}: The final exam will be held on Tuesday, 18. December, 2018, 10:00am - 11:50am.  It is cumulative. Any student who fails to earn 60 \% or better on the final exam will receive a grade no higher than a D+.

\vskip.25in
\noindent\textbf{Academic Honesty}:  All students are expected to adhere to the SMSU Code of Conduct. Any type of academic dishonesty will not be tolerated. In the event of academic dishonesty, the student will receive an F in the course.

\vskip.25in
\noindent\textbf{Extra Help}:  Do not hesitate to come to my office during office hours or by appointment. The SMSU Math Help Center, which offers free tutoring, is also available to you. The Math Help Center is located in the Academic Commons, IL224.


\vskip.25in
\noindent\textbf{Attendance Policy}: Students are expected to attend classes regularly.  Attendance will occasionally be taken. Should you miss class, it is your responsibility to obtain a copy of the notes from a classmate. You are responsible for the material that was covered and any announcements that were made.

\vskip.25in
\noindent\textbf{Calculators}: Calculators may not be used on exams or on worksheets.

\vskip.25in
\noindent\textbf{D2L}: Announcements and updates to this syllabus can be found on our course's D2L website.


\vskip.25in
\noindent\textbf{Important Dates}:
\begin{center} \begin{minipage}{5in}
\begin{flushleft}
Last date to drop with full refund 	\dotfill 31. August \\
Last date to drop with a W 			\dotfill 27. November\\
No class (Labor Day)				\dotfill 03. September\\
No class (Fall Break)               \dotfill 15-16. October \\
No class (Veterans Day)				\dotfill 12. November\\ 
No class (Thanksgiving)				\dotfill 21-23. November\\
Final Exam 							\dotfill 18. December, 2016, 10:00am - 11:50am\\
\end{flushleft}
\end{minipage}
\end{center}








\end{document}